\documentclass{article} 
\usepackage{comment}
\usepackage{multicol}
\usepackage{hyperref}
%\usepackage[a4paper, margin=2.5cm]{geometry}
\usepackage{geometry}
\geometry{a4paper,left=.8in,right=.8in,top=.8in,bottom=.8in}
\usepackage{cite}
\usepackage[utf8]{inputenc}
%%%%%%%%%%%%%%%%%%%%
\renewcommand{\thesection}{\Alph{section}}
\newcommand{\papertitle}[1]{\textbf{ ``#1''}}
\newcommand{\authorMe}{{\bf Subhrendu Chattopadhyay}}
\newcommand{\name}{\expandafter\MakeUppercase\expandafter{Subhrendu Chattopadhyay}}
\renewcommand\refname{List of Publications}
%%%%%%%%%%%%%%%%%%%%
%\setlength{\textheight}{9.5in} % increase text height to fit on 1-page 

\begin{document} 
\begin{center}
	\Large{\bf \name}
\end{center}
 \noindent\makebox[\linewidth]{\rule{\textwidth}{0.4pt}}
\section{Contact Information}
\setlength{\columnsep}{-2.5cm}

\begin{multicols}{2}
	\begin{itemize}
		 \item{\bf Present Address:}\\
		 365, Maharaja Yadvindra Enclave\\
		 Nabha Road, Patiala\\
		 Punjub, India 147005
		\item{\bf Permanent Address:}\\
		 c/o Subhas Ch. Chattopadhyay,\\
		 55-Charichara Bazar Lane, Nabadwip,\\
		 Nadia, Westbengal, India 741302
	\columnbreak
		\item {\bf Website:} \url{https://subhrendu1987.github.io/}
		\item {\bf Email:} \url{subhrendu.subho@gmail.com}
		\item {\bf Mobile:} +91-9435~658~234, +91-8473~894~164
		\item {\bf Skype:} live:subhrendu.subho\_1
		\item {\bf GitHub:} \url{https://github.com/subhrendu1987}
	\end{itemize}
\end{multicols}
\section{Research Objective}
\par At present I am an Assistant Professor in the Department of Computer Science and Engineering, Thapar Institute of Engineering and Technology (TIET).

My reserch interests are in computer systems (e.g. Computer Networks, Operating Systems, Distributed Systems). During my PhD my primary research focus was in Software Defined Networking (SDN), Network Function Virtualization (NFV), Fog Computing, Next Generation Networks, and Performance Modeling of Network and Communication Systems. My contributions were to enhance the scalability of SDN in large-scale IoT environments, where I successfully developed and implemented orchestration frameworks that automate deployment and provide robust fault and partition tolerance. I am passionate about creating self-managing, future-proof network architectures.

At IDRBT, I was instrumental in the 5G Use-Case Lab for BFSI, contributing to the identification and development of critical India-specific 5G applications for the financial sector. I also played a key role in the establishment of the Network Innovation Lab (NIL), a dedicated facility for advancing network design, development, and management to foster evolutionary network architectures and test financial applications. 

In my current assignment I have continued to am actively working on 
In both capacities, I engaged in in-depth research at the intersection of network, operating systems and distributed systems.

I like to diagnose system problems, simplifying them to their essence, and conceptualizing elegant solutions. This analytical approach clarifies issues and often reveals interdisciplinary connections, allowing me to leverage diverse insights for innovative problem-solving.
\section{Academic Qualification} 
	\begin{itemize}
		\item{\bf PhD:} Doctor  of Philosophy in Computer Science and Engineering from Indian Institute of Technology, Guwahati (July,2014 - April,2021)
		\item{\bf Post Graduation:} Master of Technology in Computer Science and Engineering with {\bf CGPA: 8.81/10} from Indian Institute of Technology, Guwahati (June,2012 - July,2014)
		\item{\bf Graduation:}Bachelor of Technology in Computer Science and Engineering with {\bf CGPA: 8.04/10} from B.P Poddar Institute of Management and Technology, WestBengal University of Technology (July,2006 - June,2010)
		\item{\bf Higher Secondary (10+2):} with {\bf 77.5\%} from Beldanga C.R.G.S High School, under West Bengal Council of Higher Secondary Examination (May,2006)
		\item{\bf Secondary (10):} Madhyamik with {\bf 81.5\%} from Sargachhi Ramakarishna Mission High School, under West Bengal Board of Secondary Education (April,2003)
	\end{itemize}

\section{Professional Experience}
	\begin{itemize}
		\item{\bf Assistant Professor:} Thapar Institute of Engineering and Technology (TIET), Patiala(January,2025 - Till Date)
		\item{\bf Assistant Professor:} Institute for Development and Research in Banking Technology (IDRBT), Hyderabad(April,2022 - December,2024)
		\item{\bf Assistant Professor:} Department of CSE in SRM-University, AP(June 2021 - April 2022)
		\item{\bf Temporary Project Staff:} Department of Computer Science and Engineering in Indian Institute of Technology, Kharagpur (October,2020 - March,2021)\\
	Project Name: Development of Algorithms and Tools for Log Analytics  and Vulnerability Assessment \\
	Principal Investigator: Dr. Sandip Chakraborty
		\item{\bf Automation Test Engineer:} Programmer Analyst Trainee in Cognizant Technology Solution India Pvt. Ltd. (July,2010 - July,2011)
	\end{itemize}

\section{Thesis}
Subhrendu Chattopadhyay,{\em SDN for Large Scale IoT Networks}, PhD thesis, Supervised by Prof. Sukumar Nandi, Indian Institute of Technology Guwahati, \url{http://gyan.iitg.ernet.in/handle/123456789/1854}, 2021.
\section{Awards}
	\begin{enumerate}
		\item{\bf Fellowship:} Recipient of TCS Research scholarship (Cycle 10) and Fellowship from MHRD
		\item {\bf Travel Grants:} 
			\begin{enumerate}
				\item Received conference travel grant from IEEE COMSNETS and LRN foundation.
				\item Recipient of travel grant from Microsoft India, Research and Development
			\end{enumerate}
		\item {\bf Best paper awards:} 
			\begin{enumerate}
				\item IEEE INFOCOM 2019~\cite{chattopadhyay2018aloe} (in a session)
				\item IEEE COMSNETS 2016~\cite{chakraborty2016es2}
				\item IEEE ANTS 2013~\cite{chakraborty2013surpassing}
			\end{enumerate}
	\end{enumerate}
\section{Subjects Taught}	
	\begin{enumerate}
	 \item {\bf Objected Oriented Programming with C++ (UG: Th+Lab)} in TIET
	 \item {\bf Microprocessor Based System Design (UG: Th+Lab)} in TIET
	 \item {\bf Advanced Operating Systems (PG: Th+Lab)} in IDRBT, UoH Campus
	 \item {\bf Internet Technology (PGDBT: Th+Lab)} in IDRBT
	 \item {\bf Computer Networking (UG: Th+Lab)} in SRM-University, AP
	 \item {\bf Objected Oriented Programming with C++ (UG: Th+Lab)} in SRM-University, AP	 
	 \item {\bf Operating Systems (UG: Th+Lab)} in SRM-University, AP
	\end{enumerate}
\begin{comment}
\section{Teaching Assistance}	
	\begin{enumerate}
	 \item {\bf Teaching Assistant} in IIT, Guwahati For Operating Systems (CS341) (2018 Monsoon)
	 \item {\bf Teaching Assistant} in IIT, Guwahati For Network Lab (CS343) (2016 Monsoon)
	 \item {\bf Teaching Assistant} in IIT, Guwahati For Wireless Networks (CS551) (2015 Monsoon, 2017 Monsoon)
	 \item {\bf Teaching Assistant} in IIT, Guwahati For Systems Lab (CS558) (2014 Winter,2015 Winter,2016 Winter,2018 Winter)
	 \item {\bf Teaching Assistant} in IIT, Guwahati For Programming Lab (CS513) (2013 Monsoon, 2014 Monsoon)
	 \item {\bf Teaching Assistant} in IIT, Guwahati For Computing Laboratory (CS110) (2013 Winter)
	 \item {\bf Teaching Assistant} in IIT, Guwahati For Discrete Mathematics (CS202)  (2012 Monsoon)
	\end{enumerate}
\end{comment}
\section{Projects}
	\begin{enumerate}
	 \item {\bf 5G Usecase Lab for BFSi:} Funded by DoT and DFS (GoI), PI: Prof. V.N. Sastry, Co-PI: Subhrendu Chattopadhyay, Abhishek Thakur, INR. 1.05 Cr.
	 \item {\bf Consultancy on Comprehensive IT Infrastructure Review:} A Public sector Bank, Co-PI: Subhrendu Chattopadhyay, Radha V, N.P. Dhavale, Dipanjan Roy, INR. 28,500,00
 	 \item {\bf Consultancy on SIEM Solutions:} A Private sector Bank, Co-PI: Subhrendu Chattopadhyay, Dipanjan Roy, INR. 3,00,000 
	\end{enumerate}
\section{Voluntary Services}
  \begin{enumerate}
   \item Conference Reviewer: IEEE ANTS (2014 - 2024), IEEE ICC 2017, IEEE NCC 2017, IEEE ISED 2017, IEEE COMSNETS (2018-2024), COMSYS 2023
   \item Journal Reviewer: Springer Journal of Network and Systems Management
   \item Member of Technical Program Committee: IEEE  COMSNETS (2020-2024), CSI 2022, NCC 2021, 2022, ICDCN 2023
  \end{enumerate}
\section{Collaborations}
I had the opportunity to collaborate with the following distinguished researchers.
  \begin{enumerate}
  	\item Dr. Sandip Chakraborty, Associate Professor, IIT Kharagpur [On-going]
	\item Prof. Soumya K Ghosh, Professor, IIT Kharagpur
	\item Prof. Sushanta Karmakar, Professor, IIT Guwahati
	\item Dr. Samar Shailendra, Adjunct Faculty, IIIT Bangalore
	\item Dr. Abhinandan S. Prasad, Assistant Professor, IIT Ropar [On-going]
	\item Dr. Niladri Sett, Assistant Professor, SRM University AP
	\item Dr. Soumyajit Chatterjee, Research Scientist, Nokia Bell Labs Cambridge
	\item Dr. Shubhabrata Nath, Assistant Professor, IIIT Guwahati
	%\item Dr. Kotaro Kataoka, Associate Professor, IIT Hyderabad [On-going]
  \end{enumerate}
\section{Reference Persons}
  \begin{enumerate}
  \item Prof. Sukumar Nandi, Senior Professor\\Department of CSE, IIT Guwahati, Assam, India-781039, sukumar@iitg.ac.in, (+91~361~258~2357)
  \item Dr. Sandip Chakraborty, Associate Professor\\Department of CSE, IIT Kharagpur, West Bengal, India-721302, sandipc@cse.iitkgp.ac.in, (+91~322~228~2898)
  \item Prof. V. N. Sastry, Professor\\IDRBT, Masab Tank, Hyderabad, Telengana, India-500057, vnsastry@idrbt.ac.in, (+91~40~2329~4304)
  \item Prof. Soumya Kanti Ghosh, Professor\\Department of CSE, IIT Kharagpur, West Bengal, India-721302, skg@cse.iitkgp.ac.in, (+91~322~228~2332)
  \end{enumerate}
%%%%%%%%%%%%%%%%%%%%%%%%%%%%%%%%%%%%%%%%%%%%%%%%%%%%%%%%%%%%
\phantom{
\cite{mondal_granulotrack_underreview},
\cite{satapathy_xplog_underreview},
\cite{dutta2025autopac},
\cite{chowdhary2025beeguard}
\cite{tripathi2024made},
\cite{satapathy2023disprotrack},
\cite{nath2021containerized},
\cite{chattopadhyay2020aloe},
\cite{chattopadhyay2020amalgam},
\cite{nath2019ptc},
\cite{chattopadhyay2018improving},
\cite{chattopadhyay2018aloe},
\cite{pranav2017fast},
\cite{chattopadhyay2017primary},
\cite{chattopadhyay2017flipper},
\cite{chakraborty2016alleviating},
\cite{chakraborty2016es2},
\cite{chattopadhayay2016leveraging},
\cite{sett2016time},
\cite{chakraborty2014defending},
\cite{karmakar2014trigger},
\cite{chakraborty2013surpassing}}
%%%%%%%%%%%%%%%%%%%%%%%%%%%%%%%%%%%%%%%%%%%%%%%%%%%%%%%%%%%%
  \bibliographystyle{plainyr-rev}
  %\bibliographystyle{../common/ieee/IEEEtran}
  %\bibliographystyle{unsrt}
  \bibliography{../common/20_bibilography/mypub}
%\end{resume}

\end{document}
\documentclass{article} 
\usepackage{comment}
\usepackage{multicol}
\usepackage{hyperref}
%\usepackage[a4paper, margin=2.5cm]{geometry}
\usepackage{geometry}
\geometry{a4paper,left=.8in,right=.8in,top=.8in,bottom=.8in}
\usepackage{cite}
\usepackage[utf8]{inputenc}
%%%%%%%%%%%%%%%%%%%%
\renewcommand{\thesection}{\Alph{section}}
\newcommand{\papertitle}[1]{\textbf{ ``#1''}}
\newcommand{\authorMe}{{\bf Subhrendu Chattopadhyay}}
\newcommand{\name}{\expandafter\MakeUppercase\expandafter{Subhrendu Chattopadhyay}}
\renewcommand\refname{List of Publications}
%%%%%%%%%%%%%%%%%%%%
%\setlength{\textheight}{9.5in} % increase text height to fit on 1-page 

\begin{document} 
\begin{center}
	\Large{\bf \name}
\end{center}
 \noindent\makebox[\linewidth]{\rule{\textwidth}{0.4pt}}
\section{Contact Information}
\setlength{\columnsep}{-2.5cm}

\begin{multicols}{2}
	\begin{itemize}
		 \item{\bf Present Address:}\\
		 L-202, IDRBT/RBI Staff Quarters\\
		 5, Lane Number 1, Begumpet, Hyderabad\\
		 Telangana, India 500016
		\item{\bf Permanent Address:}\\
		 c/o Subhas Ch. Chattopadhyay,\\
		 55-Charichara Bazar Lane, Nabadwip,\\
		 Nadia, Westbengal, India 741302
	\columnbreak
		\item {\bf Website:} \url{https://subhrendu1987.github.io/}
		\item {\bf Email:} \url{subhrendu.subho@gmail.com}
		\item {\bf Mobile:} +91-9435~658~234, +91-8473~894~164
		\item {\bf Skype:} live:subhrendu.subho\_1
		\item {\bf GitHub:} \url{https://github.com/subhrendu1987}
	\end{itemize}
\end{multicols}
\section{Research Objective}
	My research interests include Software Defined Networking (SDN), Network Function Virtualization (NFV), Fog Computing, Next Generation Networks and Performance Modeling of Network and Communication System. During my PhD, I have worked on the scalability issues that arise during deployment of SDN to provide network management to the large scale networks consisting of Internet of things (IoT) devices. I have developed multiple ``orchestration" frameworks that automate the deployment challenges and provide fault and partition tolerance to the system. I am particularly interested in developing a management-free and future-proof network architecture. 

Currently, at IDRBT I am part of 5G Use-Case Lab for BFSi where we are trying to identify and develop India-specific use cases of 5G in Banking and Financial Services industries (BFSi). Additionally, I am also trying to set-up Network Innovation Lab (NIL) which will be dedicated to design, development, and management of different types of network architectures in order to achieve an evolutionary network design and a prototype infrastructure to test the behaviour of newly developed financial applications. In both labs we are investigating the research problems and challenges related to network and operating systems.

My approach is to identify a systems problem, reduce it to its core by stripping away unnecessary details, and look for clean conceptual solutions. Such an approach often clarifies the issue and shows possible connections to other areas where we can borrow ideas and develop innovative solutions.
\section{Academic Qualification} 
	\begin{itemize}
		\item{\bf Post Graduation:} Doctor  of Philosophy in Computer Science and Engineering from Indian Institute of Technology, Guwahati (July,2014 - April,2021)
		\item{\bf Post Graduation:} Master of Technology in Computer Science and Engineering with {\bf CGPA: 8.81/10} from Indian Institute of Technology, Guwahati (June,2012 - July,2014)
		\item{\bf Graduation:}Bachelor of Technology in Computer Science and Engineering with {\bf CGPA: 8.04/10} from B.P Poddar Institute of Management and Technology, WestBengal University of Technology (July,2006 - June,2010)
		\item{\bf Higher Secondary (10+2):} with {\bf 77.5\%} from Beldanga C.R.G.S High School, under West Bengal Council of Higher Secondary Examination (May,2006)
		\item{\bf Secondary (10):} Madhyamik with {\bf 81.5\%} from Sargachhi Ramakarishna Mission High School, under West Bengal Board of Secondary Education (April,2003)
	\end{itemize}

\section{Professional Experience}
	\begin{itemize}
		\item{\bf Assistant Professor:} Institute for Development and Research in Banking Technology (IDRBT), Hyderabad(April,2022 - Till date)
		\item{\bf Assistant Professor:} Department of CSE in SRM-University, AP(June 2021 - April 2022)
		\item{\bf Temporary Project Staff:} Department of Computer Science and Engineering in Indian Institute of Technology, Kharagpur (October,2020 - March,2021)\\
	Project Name: Development of Algorithms and Tools for Log Analytics  and Vulnerability Assessment \\
	Principal Investigator: Dr. Sandip Chakraborty
		\item{\bf Automation Test Engineer:} Programmer Analyst Trainee in Cognizant Technology Solution India Pvt. Ltd. (July,2010 - July,2011)
	\end{itemize}

\section{Thesis}
Subhrendu Chattopadhyay,{\em SDN for Large Scale IoT Networks}, PhD thesis, Supervised by Prof. Sukumar Nandi, Indian Institute of Technology Guwahati, \url{http://gyan.iitg.ernet.in/handle/123456789/1854}, 2021.
\section{Awards}
	\begin{enumerate}
		\item{\bf Fellowship:} Recipient of TCS Research scholarship (Cycle 10) and Fellowship from MHRD
		\item {\bf Travel Grants:} 
			\begin{enumerate}
				\item Received conference travel grant from IEEE COMSNETS and LRN foundation.
				\item Recipient of travel grant from Mirosoft India, Research and Development
			\end{enumerate}
		\item {\bf Best paper awards:} 
			\begin{enumerate}
				\item IEEE INFOCOM 2019~\cite{chattopadhyay2018aloe} (in a session)
				\item IEEE COMSNETS 2016~\cite{chakraborty2016es2}
				\item IEEE ANTS 2013~\cite{chakraborty2013surpassing}
			\end{enumerate}
	\end{enumerate}
\section{Subjects Taught}	
	\begin{enumerate}
	 \item {\bf Advanced Operating Systems (PG: Th+Lab)} in IDRBT, UoH Campus
	 \item {\bf Internet Technology (PGDBT: Th+Lab)} in IDRBT
	 \item {\bf Computer Networking (UG: Th+Lab)} in SRM-University, AP
	 \item {\bf Objected Oriented Programming with C++ (UG: Th+Lab)} in SRM-University, AP	 
	 \item {\bf Operating Systems (UG: Th+Lab)} in SRM-University, AP
	\end{enumerate}
\begin{comment}
\section{Teaching Assistance}	
	\begin{enumerate}
	 \item {\bf Teaching Assistant} in IIT, Guwahati For Operating Systems (CS341) (2018 Monsoon)
	 \item {\bf Teaching Assistant} in IIT, Guwahati For Network Lab (CS343) (2016 Monsoon)
	 \item {\bf Teaching Assistant} in IIT, Guwahati For Wireless Networks (CS551) (2015 Monsoon, 2017 Monsoon)
	 \item {\bf Teaching Assistant} in IIT, Guwahati For Systems Lab (CS558) (2014 Winter,2015 Winter,2016 Winter,2018 Winter)
	 \item {\bf Teaching Assistant} in IIT, Guwahati For Programming Lab (CS513) (2013 Monsoon, 2014 Monsoon)
	 \item {\bf Teaching Assistant} in IIT, Guwahati For Computing Laboratory (CS110) (2013 Winter)
	 \item {\bf Teaching Assistant} in IIT, Guwahati For Discrete Mathematics (CS202)  (2012 Monsoon)
	\end{enumerate}
\end{comment}
\section{Voluntary Services}
  \begin{enumerate}
   \item Conference Reviewer: IEEE ANTS (2014 - 2018), IEEE ICC 2017, IEEE NCC 2017, IEEE ISED 2017, IEEE COMSNETS (2018-2019), COMSYS 2023
   \item Journal Reviewer: Springer Journal of Network and Systems Management
   \item Member of Technical Program Committee: IEEE  COMSNETS (2020-2024), CSI 2022, NCC 2021, 2022, ICDCN 2023
  \end{enumerate}
\section{Collaborations}
I had the opportunity to collaborate with the following distinguished researchers.
  \begin{enumerate}
  	\item Dr. Sandip Chakraborty, Associate Professor, IIT Kharagpur [On-going]
	\item Prof. Soumya K Ghosh, Professor, IIT Kharagpur
	\item Prof. Sushanta Karmakar, Professor, IIT Guwahati
	\item Dr. Samar Shailendra, Adjunct Faculty, IIIT Bangalore
	\item Dr. Abhinandan S. Prasad, Associate Professor, NIE Mysore
	\item Dr. Niladri Sett, Assistant Professor, SRM University AP
	\item Dr. Soumyajit Chatterjee, Research Scientist, Nokia Bell Labs Cambridge
	%\item Dr. Kotaro Kataoka, Associate Professor, IIT Hyderabad [On-going]
  \end{enumerate}
\section{Reference Persons}
  \begin{enumerate}
  \item Prof. Sukumar Nandi, Senior Professor\\Department of CSE, IIT Guwahati, Assam, India-781039, sukumar@iitg.ac.in, (+91~361~258~2357)
  \item Dr. Sandip Chakraborty, Associate Professor\\Department of CSE, IIT Kharagpur, West Bengal, India-721302, sandipc@cse.iitkgp.ac.in, (+91~322~228~2898)
  \item Prof. V. N. Sastry, Professor\\IDRBT, Masab Tank, Hyderabad, Telengana, India-500057, vnsastry@idrbt.ac.in, (+91~40~2329~4304)
  \item Prof. Soumya Kanti Ghosh, Professor\\Department of CSE, IIT Kharagpur, West Bengal, India-721302, skg@cse.iitkgp.ac.in, (+91~322~228~2332)
  \end{enumerate}
%%%%%%%%%%%%%%%%%%%%%%%%%%%%%%%%%%%%%%%%%%%%%%%%%%%%%%%%%%%%
\phantom{
\cite{satapathy2023disprotrack},
\cite{nath2021containerized},
\cite{chattopadhyay2020aloe},
\cite{chattopadhyay2020amalgam},
\cite{nath2019ptc},
\cite{chattopadhyay2018improving},
\cite{chattopadhyay2018aloe},
\cite{pranav2017fast},
\cite{chattopadhyay2017primary},
\cite{chattopadhyay2017flipper},
\cite{chakraborty2016alleviating},
\cite{chakraborty2016es2},
\cite{chattopadhayay2016leveraging},
\cite{sett2016time},
\cite{chakraborty2014defending},
\cite{karmakar2014trigger},
\cite{chakraborty2013surpassing}}
%%%%%%%%%%%%%%%%%%%%%%%%%%%%%%%%%%%%%%%%%%%%%%%%%%%%%%%%%%%%
  \bibliographystyle{plainyr-rev}
  %\bibliographystyle{../common/ieee/IEEEtran}
  %\bibliographystyle{unsrt}
  \bibliography{../common/20_bibilography/mypub}
%\end{resume}

\end{document}
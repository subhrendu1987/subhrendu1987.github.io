\addac{IoT} is one of the rapidly growing network technologies which has the potential to serve millions of devices. Such \addac{LSN} requires network management to efficiently serve the end-user applications. The modern network management systems are expected to identify the time varying traffic pattern and take suitable actions to ensure fine grained network management. Taking this kind of dynamic decisions requires programmability in the network, where the programmable network management system can be used to deploy evolutionary protocols rapidly based on the objective. However, traditional Internet architecture suffers from lack of flexibility due to absence of programmability.  \addac{SDN} have emerged to provide a flexible architecture for network control and management. Additionally \addac{SDN} provides opportunities to cater ever increasing bandwidth demand by fine tuning the network resources. The objective of this thesis is to design a distributed scalable \addac{SDN} orchestration framework which is suitable for handling the dynamic nature of \addac{LSN}. In this thesis we explore the performance improvement of \addac{IoT} applications by utilizing \addac{SDN}. Subsequently we explore various deployment and architectural design related issues of \addac{SDN}.

Modern \addac{IoT} and hand-held devices are equipped with multiple interfaces. To leverage the bandwidth capacity of multiple interfaces several multi-path transport layer protocols exist which provides bandwidth aggregation. The first contribution in this thesis enhances the performance of \addac{IoT} applications by proposing \addac{SDN} control plane application \paperName{SDN-MPTCP} for \addac{MPTCP}, where \addac{MPTCP} is one of the popular multipath transport protocol. In this work we find that, the performance of \addac{MPTCP} has a strong correlation with the selected paths, and \addac{SDN} can assist in path selection of \addac{MPTCP}. During the performance improvement by employing \addac{SDN} we understood that, the deployment of \addac{SDN} over an existing \addac{LSN} increase the \addac{capex} of the system. Moreover the centralized nature of \addac{SDN} control plane becomes single point of failure for the network operation. Therefore, in this next work we investigate the \addac{SDN} deployment challenges for \addac{LSN}. As mentioned earlier, \addac{SDN} requires deployment of \addac{SDN} supported hardware, which In this work we utilized \addac{NFV} for development of \paperName{FLIPPER}. \paperName{FLIPPER} enables deployment of \addac{SDN} like network management over existing \addac{COTS} devices of \addac{LSN} by converting them into \addac{PDEP}. \paperName{FLIPPER} provides a scalable, flexible, fail-safe and distributed \addTerm{self-stabilized} architecture. In the next contribution we use \paperName{FLIPPER} to design \paperName{Aloe} orchestration framework which utilizes the \addac{INP} platforms of \addac{LSN} to achieve \addTerm{servicification} of \addac{SDN} control plane. \paperName{Aloe} promises \addTerm{plug-and-play} and \addTerm{zero touch deployment} along with light-weight, fault-tolerant and auto-scalable network management platform for \addac{LSN}. Through exhaustive experimentation over an in-house test bed and \addac{AWS} platform we find that, \paperName{Aloe} can significantly improve performance of various \addac{IoT} applications. During this study we also observed that, various end-user applications targeted for \addac{LSN} require \addac{VNF} based \addac{SFC} depending on the network service access policy. However, dynamic deployment of \addacp{VNF} and traffic steering through those \addacp{VNF} to preserve the \addac{SFC} ordering is difficult in a \addac{LSN} which spans across multiple administrative domain. In the next contribution we propose \paperName{Amalgam} which incorporates \addac{SFC} management with \paperName{Aloe} to ensure scalability and dynamic \addac{SFC}. Based on the NP-hard nature of \addac{VNF} placement problem, \paperName{Amalgam} also proposes a greedy heuristic for \addac{VNF} placement which ensures fast flow initialization and provides performance improvement for short-duration flows. As a whole, this thesis provides auto-scalable and fault-tolerant distributed architecture and orchestration framework for \addac{LSN} network management to provide fine-grained network control. We have compared the proposed solutions with the state-of-the-art works. Based on the experimental results we found that, the proposed solutions can provide significant performance improvement for short duration flows while incurring lower resource consumption of the system.
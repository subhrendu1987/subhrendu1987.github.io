In this thesis, we have investigated scalability and network management issues of large scale \addac{IoT} network (\addac{LSN}) by utilizing \addac{SDN}. Apart from standard \addac{IoT} features \addac{LSN} supports \addac{INP} and spans across multiple administrative domains. Scale of the network and use of resource constrained \addac{COTS} devices makes the network unstable and prone to failures. The proposed techniques and architectures in this thesis are designed to provide failure tolerance. During design of the architectures we have also considered traffic characteristics of \addac{LSN} to improve the end-to-end network performance. The major contributions of this thesis can be summarized as follows.

The use of \addac{SDN} for network management has been well studied in the literature. We observed that the network management decisions like route discovery have a strong correlation with the transport layer decision like path management. This correlation increases in the presence of a multi-path transport protocol. In \cref{chapter2}, we developed \paperName{SDN-MPTCP}, which helps the transport layer decision making of \addac{MPTCP}, a popular multi-path transport layer protocol, by exploiting \addac{SDN}. We observed that the proposed \paperName{SDN-MPTCP} can reduce the \addac{HOL} blocking problem of \addac{MPTCP} sub-flow selection.

Even though, \addac{SDN} can improve performance of \addac{LSN}, deployment of \addac{SDN} over an existing \addac{LSN} is challenging. The \addac{capex}/\addac{opex} introduced at the time of deployment of \addac{SDN} over a \addac{LSN} environment restricts the system managers to adopt \addac{SDN} for network management. To overcome this issue, we propose \paperName{FLIPPER} in \cref{chapter3}. \paperName{FLIPPER} lies some where between the traditional network architecture and \addac{SDN} and provides \addac{SDN} like control of the \addac{COTS} devices of the \addac{LSN}. Additionally \paperName{FLIPPER} can provide fault-tolerance by dynamically assigning roles using \addac{NFV}. \paperName{FLIPPER} can reduce the flow initiation delay by dynamically deploying \addac{NIB}.

One of the major challenges in \addac{SDN} based management of \addac{LSN} lies in its dynamic nature, where the devices can enter/leave the eco-system without notification. In \cref{chapter4} we extended the \paperName{FLIPPER} to provide plug-and-play support to tackle the dynamic behaviour of \addac{LSN} components by developing \paperName{Aloe}. \paperName{Aloe} provides zero-touch deployment with self-healing properties through self-stabilization. \paperName{Aloe} proposes servicification of control plane over the \addac{INP} platform provided by \addac{LSN}. We also proposed a light-weight controller independent control plane framework to enhance the capabilities of \paperName{Aloe}.

Apart from core services like routing, \addac{QoS} etc., management of a large scale network like \addac{LSN} requires multiple auxiliary \addac{VNF} services like \addac{NAT}, proxy etc. Where the core functionalities of \addac{SDN} are supported in \paperName{Aloe}, the existence of \addac{VNF} requires more attention. In \cref{chapter5} we propose \paperName{Amalgam} which proposes the integration of middle-box application into \paperName{Aloe}. Like \paperName{Aloe}, \paperName{Amalgam} provides plug-and-play support for \addac{LSN} which is spanned over multiple administrative domain. Additionally, \paperName{Amalgam} proposes a novel \addac{VNF} deployment and traffic steering framework to support \addac{SFC} over \addac{LSN}. For experimental purpose, we developed a new emulation tool \paperName{MiniDockNet} which overcomes the issue of \addac{INP} emulation using existing \addTerm{Mininet} emulation framework. Based on the experimental results we found that, \paperName{Amalgam} performs well for short flows.

\subsection{Future Direction}
The management capabilities and performance of \addac{LSN} can further be improved by providing support for advanced features to the proposed orchestration frameworks and architectures which are kept as the future direction of this thesis. Some of the features are discussed as follows.
\subsubsection{Enhancement of Amalgam}
In this thesis, the proposed \paperName{Amalgam} uses a distributed greedy heuristics for the placement of \addac{VNF}. The proposed \addac{VNF} placement strategy provides initial benefits to the short-duration flows. However, there is room for improvement here. A pro-active deployment of \acp{VNF} can improve the performance for short as well as long-duration flows. On the other hand, pro-active placement of \acp{VNF} requires the prediction of upcoming traffic requirements and load distribution of the devices. We have some initial experimental results~\cite{nath2019ptc:own} to believe that an online reinforcement learning mechanism can help in this direction, which we kept as a future work of this research.
\subsubsection{Dynamic Telemetric Application Deployment}
The proposed \paperName{Aloe} orchestration framework relies on the dynamic deployment of control plane applications based on the necessity (in our case failure events). However, this feature can be customized for multiple other event monitoring purposes. For example, a flow monitor can be auto-inserted into a particular region of the \addac{LSN} to estimate security lapses, traffic pattern analysis, identification of a heavy hitter flow etc. based on a system administrator defined policy. We kept this customizability as a future work for \paperName{Aloe}.
